\documentclass{article}
\usepackage{fancyhdr}
\usepackage[margin=1in]{geometry}

\pagestyle{fancy}
\lhead{Pravin Ramana}
\rhead{CS3341 Foundations of Modern Computing Fall 2023}
\renewcommand{\headrulewidth}{0pt}

\begin{document}
\section*{Multithreading Job System Report}
	\subsection{Memory Management and Thread Protection Paradigms}
	The architecture of the job system is similar to the one detailed in class, however it makes several different design decisions to reduce complexity.
		\subsubsection{Guards over raw mutexes}
		One of the first is by using \texttt{guard\_lock}s or \texttt{unique\_lock}s, when necessary. The benefit to using these is that in the event of an exception, locking will occuring automatically when the locks go out of scope (or when explicitly unlocked), reducing the chances of unintended deadlocks. These types of mutex come at the disadvantage that they provide some overhead, however, it is small enough that it is work the additional safety that they provide.
		\subsubsection{Message Passing}
			One of the other notable changes is the use of a message passing system backed by kernel interrupts, rather than busy-waiting. In general, spinning in userspace is a bad idea, as it may compete for CPU time with the OS, or introduce delays. To avoid this, a \texttt{MessageQueue} system is used for passing data between the Master thread and its Slave threads. This allows threads to efficiently sleep while waiting for work, rather than waking up at an interval to check for work. In general, this solution is advantageous, but may cause slowdowns when context-switching is expensive.
		\subsubsection{No public system-scoped memory}
			The \texttt{JobSystem} itself is designed such that it will not carry any publicly accessible memory. All memory that is publicly accessible, including memory allocated by jobs themselves, is managed under the responsibility of the programmer. All jobs that are handled by the system will have memory that may outlive the scope of the \texttt{JobSystem} and can be freed by the programmer at his free-will. The effect of this is that complexity that might've existed within the system is now a burden to the programmer.
\end{document}
