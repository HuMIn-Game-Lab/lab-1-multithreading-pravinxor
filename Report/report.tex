\documentclass{article}
\usepackage{fancyhdr}
\usepackage[margin=1in]{geometry}
\usepackage{svg}

\pagestyle{fancy}
\lhead{Pravin Ramana}
\rhead{CS3341 Foundations of Modern Computing Fall 2023}
\renewcommand{\headrulewidth}{0pt}

\begin{document}
\section*{Multithreading Job System Report}
	\subsection{Memory Management and Thread Protection Paradigms}
	The architecture of the job system is similar to the one detailed in class, however it makes several different design decisions to reduce complexity.
		\subsubsection{Guards over raw mutexes}
		One of the first changes is by using scope-based \texttt{lock\_guard}s or \texttt{unique\_lock}s, rather than raw mutexes. The benefit to using these is that in the event of an exception, locking will occur automatically when the guard goes out of scope (or when explicitly unlocked), reducing the chances of unintended deadlocks. This, however, come at the disadvantage that they add some overhead, however, it is small enough that it is worth the additional safety that is offered when using them.
		\subsubsection{Efficient Message Passing}
			One of the other notable changes is the use of a message passing system backed by kernel interrupts, rather than busy-waiting. In general, spinning in userspace is a bad idea, as it may compete for CPU time with the OS, or introduce delays. The effect of this is less noticeable when there are few threads, but may become more impactful when many more threads than physical cores are created. To avoid this, a \texttt{MessageQueue} is used for passing data between the Master thread and its Slave threads. This allows threads to efficiently sleep while waiting for work, rather than waking up at an interval to check for work. In general, this solution is advantageous, but may cause slowdowns when context-switching from kernelmode to usermode is expensive.
		\subsubsection{No public system-scoped memory}
			The \texttt{JobSystem} itself is designed such that it will not carry any publicly accessible memory. All memory that is publicly accessible, including memory allocated by jobs themselves, is managed under the responsibility of the programmer. All jobs that are handled by the system will have memory that may outlive the scope of the \texttt{JobSystem} and can be freed by the programmer at his free-will. In effect, this reduces complexity within the system, however it adds it as an additional burden on the programmer.
		\subsubsection{Job chaining}
			While circular dependencies are generally regarded as an antipattern, it is applicable in this scenario. Because make jobs and parsing jobs always lead to a single outcome (to generate JSON output of compiler messages), it is acceptable to allow jobs to automatically call/chain other jobs upon completion. This allows the inclusion of support for different compilers or parsing mechanisms without needing to rewrite or handle cases for function along the job pipeline. When a make-job (in this case, the root job) is called, it is always expected to end with a JSON-job output. Additionally, this also relieves the main thread of additional job management.
	\subsection{UML Overview}
		\includesvg[inkscapelatex=false,width=\textwidth]{diagram}
	\subsection{Usage of \texttt{JobSystem}}
		\subsubsection{Preamble}
			Using the JobSystem is straightforward, however it will require bookkeeping in terms of memory usage of jobs. 
		\subsubsection{Intializing the \texttt{JobSystem}}
			First, the programmer must initialize the \texttt{JobSystem}. Here we will call it \texttt{JobSystem system;}
		\subsubsection{Initializing \texttt{Slave}s}
			Additionally, in order to do work, \texttt{Slave}s must be initialized to do work. This can be done with \texttt{system.add\_slave("thread" + std::to\_string(n));} This function may be called multiple times, however the appropriate amount varies based on the nature of the work being done. Compute-bound, branchless code may prefer to limit the number of \texttt{Slaves}s to the number of physical cores, whereas I/O bound code may opt for many more threads than physical cores.
		\subsubsection{\texttt{Job} Allocation}
			Next, memory must be allocated for a job. Note: this memory, must outlive the lifetime of the JobSystem. Not doing so will result in undefined behavior. Next we will create a new MakeJob using \texttt{MakeJob *mj = new MakeJob(0, "demo");}. The arguments for this specific job are the \texttt{id} and the make \texttt{target}. 
		\subsubsection{Submitting \texttt{Job}s into the \texttt{JobSystem}}
			Next, to execute the job, we will enqueue it into the system using \texttt{system->enqueue(mj)}. The \texttt{enqueue()} function will accept any pointer, which inherits from the \texttt{Job} API. Once the \texttt{Job*} has entered the system, it will be sent to the \texttt{Slave}s in a FIFO manner, (however it is not guaranteed to be completed in a FIFO manner).
		\subsubsection{Receiving completed \texttt{Job}s from the \texttt{JobSystem}}
			The \texttt{JobSystem} offers a basic interface for recieving completed jobs. \texttt{js.get\_completed(n);} will block execution until \texttt{n} jobs are available to be recieved, and returned. Note: there is no logical checking that is done by the function itself. It is up to the programmer to determine whether it is possible for \texttt{n} jobs to be available, otherwise the program may hang if an \texttt{n} that is too large.
	
	\subsection{Demonstration of \texttt{JobSystem}}
		\subsubsection{Overview}
			The \texttt{JobSystem} implementation being demonstrated is being used to build targets from a Makefile in parallel. It is provided arguments, which are the names of the targets from a Makefile in the current directory. When the \texttt{JobSystem} is built into a binary, say named \texttt{system}.
			
		\subsubsection{Makefile targets}
			Given a Makefile with the contents:
			\begin{verbatim}
			demo: ./Code/Demo/main.cpp
				clang++ -pipe -Wall ./Code/Demo/*.cpp -o demo
			\end{verbatim}
		\subsubsection{Internal operations}
			When running \texttt{./system demo}, the system will internally call the makefile and call parsing and JSON generating jobs after. The JSON output is held in a queue until all make targets are completed. Since there is only a single target, it will be printed to stdout immediately.
		\subsubsection{Output}
			The output of the program will be a single line JSON object for each built target. Errors and warnings are not differentiated in JSON format, unless it is explicitly noted by the compiler. For example, the following error output:
			\begin{verbatim}
				clang++ -pipe -Wall ./Code/Demo/*.cpp -o demo
				./Code/Demo/main.cpp:2:1: error: unknown type name 'string'; did you mean 'std::string'?
				string dsa() {
				^~~~~~
				std::string
				/usr/bin/../lib64/gcc/x86_64-pc-linux-gnu/13.2.1/../../../../include/c++/13.2.1/bits/stringfwd.h:77:33: note: 'std::string' declared here
				  typedef basic_string<char>    string;   
				                                ^
				1 error generated.

				make: *** [Makefile:5: demo] Error 1
			\end{verbatim}
			The parsed JSON will be:
			\begin{verbatim}
				{
				  "./Code/Demo/main.cpp": [
				    {
				      "chunk": [
				        "#include <iostream>",
				        "string dsa() {",
				        "  let d = {",
				        "    das auto;"
				      ],
				      "column": 1,
				      "line": 2,
				      "message": "unknown type name 'string'; did you mean 'std::string'?"
				    },
				}
			\end{verbatim}
			In the parsed output, at most, 2 lines surrounding the line containing the error are provided, along with information about the line and column \# where the error occurred, as well as the error message emitted by the clang compiler. 
\end{document}
